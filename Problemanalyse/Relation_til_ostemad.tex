\textbf{Relation til samfund}

Udnyttelse af vindenergi har været en af de mest efterforskede områder indenfor vedvarende energi i Danmark. Det skyldes netop Danmarks suveræne geografisk placering, som har en lang kystlinje og dermed mange ideelt sted hvor vindmøllerne kan udvinde energi fra, eksempelvis vestkysten. I forhold til andre vedvarende energi, er vindmøller efterhånden en meget modne teknologi, hvor effektivitetens grad er så høj at det er realistiske at erstatte fossilbrændstof. Eksempelvis, for en alminlig solcelle ligger virkningsgrad typisk på 14-16 procent, hvorimod for en almindelig vindmølle ligger virkningsgrad hel op på ca. 30 procent.    \\
Trods af den høje virkningsgrad og udbredelsen, så har vindmøller også nogen udfordringer foran sig, en af de største problem, er nemlig begrænsning i placering af møllerne. Vindmøller forudsager masse støj til nærliggende nabolaget pga. de mange bevægende del og mekanisk bevægelse. Det er især talt om en lydforurening af lavfrekvens støj eller indfralyd. Luftens ringe absorptions evne over for lavfrekvens støj bevirker til en større spredning af støjet i forhold til højfrekvens støj. Lavfrekvens støjs egenskab gør også at den ikke blot gener udendørs, men også indendørs. Hvis støjen ligger på maksimal tilladt niveau efter den danske lovgivning, betyder det at der stadig er risiko for man bliver generet af støjen indendørs. Den elektriske effekt som skabt af møllen er proportional med skabt støjniveau, dvs. at jo større møllen er og jo mere effekt den producerer, desto mere lammer den over et større areal.  Denne egenskab skaber dermed en kæmpe begrænsning for placeringer af vindmøllerne.  Vindmøllerne bliver eksempelvis kun etableret ude på landet, på kystlinjen eller ud i havet.  \\
Et anden problem er også at vindmøllens effekt er proportional med dens størrelse. Dermed betyder det også at Hvis vi kaster blik på en hurtigløber, den mest anvendt vindmølle typer i Danmark, så kræver det at vindstyrke når et relativt højt niveau før der kan udvindes energi fra vindstrømningen. Siden vindhastighed stiger, jo længere man er fra jorden, betyder det også at de flest af vindmøllerne er nødvendigvis være høj og stor, dette vil igen lede til en begrænsning ved placering af møllerne.  \\
Efter regeringsplanen, som har sæt mål for en CO2 neutral Danmark til 2050, er det vigtigt at finde andre effektiv måde at anvend vindenergi på, som skal dække over de områder hvor konventionel vindmøller ikke er muligt at anvend. Den nye koncept, Invelox, som Amerikansk firma SheerWind står med, vil måske bryde igennem de begrænsninger som nutidige vindmøller står over for og giver udnyttelse af vindenergi et nyt perspektiv. I forhold til en konventionel vindmølle, har Invelox den fordel at, det ikke lammer lige så meget som en normal vindmølle vil gøre, pga. den mekaniske opbygning. Dens egenskab til at accelerere vindhastigheden, ved hjælp af et højere tryk, betyder også at sådan en Invelox system ikke behøver at være lige så stor og høj, som en konventionel vindmølle. Derved kan problematikken om støj og skygning af andre bygning, undgås. Det vil så giv mulighed for at indfør sådan et system ind i byerne, som eksempelvis Slages kommune prøvede at indfør ved Korsør. \\
Et andet problematik som Invelox vil være en løsning til, er de svingende produktion som bæredygtig energikilde typisk har. Invelox kan nemlig opfange vind fra 3 m/s og anvend dem til produktion af energi, hvorimod for en vindmølle kræver det en hastighed på 12-14 m/s for at producer maksimal effekt. Denne egenskab betyder at vindenergi måske kan blive til en mere kontrollerbar energikilde, end andre bærdygtigkilde, såsom solenergi.     