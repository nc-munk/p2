Ulemper ved placering af vindmøller
Vindmøllerne står overfor et problem omkring placeringen af nye vindmøller, i takt med at der bliver flere mennesker og overbefolkning bliver et problem. Det vigtige er at er i fremtiden ikke vil være plads til at have begge i en sikkerhedsmæssig afstand til hinanden. Så man må stille spørgsmålet er vindmøller en bæredygtig energikilde? Man må undersøge nye placeringsmuligheder, her er havvindmølleparker blevet det store ’go-to’. Et fænomen omkring vindmøller, som viser sig både at være en force og en ulempe er at størrelsen af vindmøllen er proportionel med den energi som vindmøllen producerer. Her bliver placering og larm igen et problem, da vindmøllerne larmer mere og skal have mere plads. Her har Sheerwind kommet med deres ide til en ny vindmølleteknologi, Invelox er ifølge deres hjemmeside den perfekte løsning, den larmer ikke, optimerer udnyttelsesgraden af kraften i vinden, og ved designet kan den placeres ovenpå evt. højhuse. I takt med at vores ’mangel’ på plads kunne man udnytte Invelox’en. Problemet med Invelox er, at teknologien ikke er færdigudviklet og der ikke er nogen af licenskøberne, som har kunne få projektet til at virke til de standarder som Sheerwind har udtalt.